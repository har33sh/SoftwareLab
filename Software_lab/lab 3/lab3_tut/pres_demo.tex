\documentclass{beamer}
\usepackage[utf8]{inputenc}
\usepackage{amsmath}
\usepackage{blkarray}
\usepackage{natbib}
\usepackage{epigraph}
\usepackage{listings}
\usepackage{csquotes}
\usepackage{float}
\usepackage{color}
\usepackage{amsmath}
\usepackage{xcolor}
\usepackage{verbatim}
\usepackage[boxed,linesnumbered]{algorithm2e}  
\usetheme{Berlin}
\title{CS699}
\subtitle{Latex Advanced}
\author{Sobha Singh \\ Ajeeta Shakeet}
\institute{ Indian Institute of Technology Bombay}
\date{\today}
\begin{document}
\begin{frame}
\titlepage
\end{frame}

\begin{frame}{Table of Contents}
    \tableofcontents
\end{frame}


\begin{frame}{Introduction}
    \section{Introduction}
    \begin{itemize}
    \item LaTeX is pronounced “lay-tech” or “lah-tech,” not “la-teks.”\\
    \item LaTeX is a document preparation system for high-quality typesetting.\\
    \item LaTeX is most often used to produce  technical or scientific documents, but it can be used for almost any form of publishing.
    \end{itemize}
\end{frame}

\begin{frame}{Overview}
    \section{Overview}
    Designed by academics and easily accommodates academic use.\\~\\
    Professionally crafted predefined layouts make a document really look as if “printed.”
\end{frame}



\begin{frame}{Conclusion}
    \section{Conclusion}
    Often times, you make a mistake when creating a document. You will notice the log file reporting a problem.\\
\end{frame}


\end{document}
