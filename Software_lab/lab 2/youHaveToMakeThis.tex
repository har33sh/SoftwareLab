%% ================================================================================
%% This LaTeX file was created by AbiWord.                                         
%% AbiWord is a free, Open Source word processor.                                  
%% More information about AbiWord is available at http://www.abisource.com/        
%% ================================================================================

\documentclass[a4paper,portrait,12pt]{article}
\usepackage[latin1]{inputenc}
\usepackage{calc}
\usepackage{setspace}
\usepackage{fixltx2e}
\usepackage{graphicx}
\usepackage{multicol}
\usepackage[normalem]{ulem}
%% Please revise the following command, if your babel
%% package does not support en-IN
\usepackage[en]{babel}
\usepackage{color}
\usepackage{hyperref}
 
\begin{document}


\begin{flushleft}
Introduction to Latex
\end{flushleft}


\begin{flushleft}
M Nagaraju and Sankalp Rangare
\end{flushleft}


\begin{flushleft}
July 27, 2016
\end{flushleft}





1





\begin{flushleft}
\newpage
Contents
\end{flushleft}


\begin{flushleft}
1 About Latex
\end{flushleft}





3





\begin{flushleft}
2 Opening and Compiling Tex Document
\end{flushleft}


\begin{flushleft}
2.1 Starting and Ending . . . . . . . . . . . . . . . . . . . . . . .
\end{flushleft}


\begin{flushleft}
2.2 Compiling the LaTeX Document . . . . . . . . . . . . . . . . .
\end{flushleft}





4


4


4





\begin{flushleft}
3 Section
\end{flushleft}





5





\begin{flushleft}
4 Cross Reference
\end{flushleft}


\begin{flushleft}
4.1 \ensuremath{\backslash}label\{key\} . . . . . . . . . . . . . . . . . . . . . . . . . . . .
\end{flushleft}


\begin{flushleft}
4.2 \ensuremath{\backslash}pageref\{key\} . . . . . . . . . . . . . . . . . . . . . . . . . . .
\end{flushleft}


\begin{flushleft}
4.3 \ensuremath{\backslash}ref\{key\} . . . . . . . . . . . . . . . . . . . . . . . . . . . . .
\end{flushleft}





6


6


7


7





\begin{flushleft}
5 Fonts
\end{flushleft}


\begin{flushleft}
5.1 Font Styles . . . . . . . . . . . . . . . . . . . . . . . . . . . .
\end{flushleft}


\begin{flushleft}
5.2 Font Sizes . . . . . . . . . . . . . . . . . . . . . . . . . . . . .
\end{flushleft}





7


7


8





\begin{flushleft}
6 Images
\end{flushleft}





8





\begin{flushleft}
7 Tables
\end{flushleft}


9


\begin{flushleft}
7.1 Multi Row Tables . . . . . . . . . . . . . . . . . . . . . . . . . 9
\end{flushleft}


\begin{flushleft}
7.2 Table of Figures . . . . . . . . . . . . . . . . . . . . . . . . . . 10
\end{flushleft}


\begin{flushleft}
8 Conclusion
\end{flushleft}





11





2





\newpage
1





\begin{flushleft}
About Latex
\end{flushleft}





\begin{flushleft}
LaTeX is a word processor and document markup language. It is distinguished from typical word processors such as Microsoft Word and Apple
\end{flushleft}


\begin{flushleft}
Pages in that the writer uses plain text as opposed to formatted text, relying
\end{flushleft}


\begin{flushleft}
on markup tagging conventions to define the general structure of a document
\end{flushleft}


\begin{flushleft}
(such as article, book, and letter), to stylise text throughout a document
\end{flushleft}


\begin{flushleft}
(such as bold and italic), and to add citations and cross-referencing. A TeX
\end{flushleft}


\begin{flushleft}
distribution such as TeXlive or MikTeX is used to produce an output file
\end{flushleft}


\begin{flushleft}
(such as PDF or DVI) suitable for printing or digital distribution.
\end{flushleft}


\begin{flushleft}
LaTeX is used for the communication and publication of scientific documents in many fields, including mathematics, physics, computer science,
\end{flushleft}


\begin{flushleft}
statistics, economics, and political science.It also has a prominent role in
\end{flushleft}


\begin{flushleft}
the preparation and publication of books and articles that contain complex
\end{flushleft}


\begin{flushleft}
multilingual materials, such as Sanskrit and Arabic. LaTeX uses the TeX
\end{flushleft}


\begin{flushleft}
typesetting program for formatting its output, and is itself written in the
\end{flushleft}


\begin{flushleft}
TeX macro language.
\end{flushleft}


\begin{flushleft}
LaTeX is widely used in academia. LaTeX can be used as a standalone
\end{flushleft}


\begin{flushleft}
document preparation system, or as an intermediate format. In the latter role, for example, it is often used as part of a pipeline for translating
\end{flushleft}


\begin{flushleft}
DocBook and other XML-based formats to PDF. The typesetting system
\end{flushleft}


\begin{flushleft}
offers programmable desktop publishing features and extensive facilities for
\end{flushleft}


\begin{flushleft}
automating most aspects of typesetting and desktop publishing, including
\end{flushleft}


\begin{flushleft}
numbering and cross-referencing of tables and figures, chapter and section
\end{flushleft}


\begin{flushleft}
headings, the inclusion of graphics, page layout, indexing and bibliographies.
\end{flushleft}


\begin{flushleft}
Like TeX, LaTeX started as a writing tool for mathematicians and computer scientists, but from early in its development it has also been taken
\end{flushleft}


\begin{flushleft}
up by scholars who needed to write documents that include complex math
\end{flushleft}


\begin{flushleft}
expressions or non-Latin scripts, such as Arabic, Sanskrit and Chinese.
\end{flushleft}


\begin{flushleft}
LaTeX is intended to provide a high-level language that accesses the
\end{flushleft}


\begin{flushleft}
power of TeX. LaTeX comprises a collection of TeX macros and a program to process LaTeX documents. Because the plain TeX formatting
\end{flushleft}


\begin{flushleft}
commands are elementary, it provides authors with ready-made commands
\end{flushleft}


3





\begin{flushleft}
\newpage
for formatting and layout requirements such as chapter headings, footnotes,
\end{flushleft}


\begin{flushleft}
cross-references and bibliographies.
\end{flushleft}


\begin{flushleft}
LaTeX was originally written in the early 1980s by Leslie Lamport at
\end{flushleft}


\begin{flushleft}
SRI International. The current version is LaTeX2e. LaTeX is free software
\end{flushleft}


\begin{flushleft}
and is distributed under the LaTeX Project Public License (LPPL) (Source
\end{flushleft}


\begin{flushleft}
Wikipedia).
\end{flushleft}





2





\begin{flushleft}
Opening and Compiling Tex Document
\end{flushleft}





\begin{flushleft}
First create a .tex file using text editor such as Vi or Gedit or Kile.
\end{flushleft}





2.1





\begin{flushleft}
Starting and Ending
\end{flushleft}





\begin{flushleft}
A minimal input file looks like following
\end{flushleft}


\begin{flushleft}
\ensuremath{\backslash}documentclass\{class\}
\end{flushleft}


\begin{flushleft}
\ensuremath{\backslash}begin\{document\}
\end{flushleft}


\begin{flushleft}
your text...
\end{flushleft}


\begin{flushleft}
\ensuremath{\backslash}end \{document\}
\end{flushleft}


\begin{flushleft}
where the class is a valid document class for LaTeX.
\end{flushleft}





2.2





\begin{flushleft}
Compiling the LaTeX Document
\end{flushleft}





\begin{flushleft}
We open the terminal and go to the directory in which our .tex file is stored
\end{flushleft}


\begin{flushleft}
and the we execute the command
\end{flushleft}


\begin{flushleft}
pdflatex example.tex
\end{flushleft}





4





\newpage
3





\begin{flushleft}
Section
\end{flushleft}





\begin{flushleft}
Sectioning commands provide the means to structure your text into units:
\end{flushleft}


\begin{flushleft}
\ensuremath{\backslash}part
\end{flushleft}


\begin{flushleft}
\ensuremath{\backslash}chapter
\end{flushleft}


\begin{flushleft}
(report and book class only)
\end{flushleft}


\begin{flushleft}
\ensuremath{\backslash}section
\end{flushleft}


\begin{flushleft}
\ensuremath{\backslash}subsection
\end{flushleft}


\begin{flushleft}
\ensuremath{\backslash}subsubsection
\end{flushleft}


\begin{flushleft}
\ensuremath{\backslash}paragraph
\end{flushleft}


\begin{flushleft}
\ensuremath{\backslash}subparagraph
\end{flushleft}


\begin{flushleft}
All sectioning commands take the same general form, e.g.,
\end{flushleft}


\begin{flushleft}
\ensuremath{\backslash}chapter[toctitle]\{title\}
\end{flushleft}


\begin{flushleft}
In addition to providing the heading title in the main text, the section
\end{flushleft}


\begin{flushleft}
title can appear in two other places:
\end{flushleft}


\begin{flushleft}
1. The table of contents.
\end{flushleft}


\begin{flushleft}
2. The running head at the top of the page.
\end{flushleft}


\begin{flushleft}
You may not want the same text in these places as in the main text. To
\end{flushleft}


\begin{flushleft}
handle this, the sectioning commands have an optional argument toctitle
\end{flushleft}


\begin{flushleft}
that, when given, specifies the text for these other places.
\end{flushleft}


\begin{flushleft}
Also, all sectioning commands have *-forms that print title as usual, but
\end{flushleft}


\begin{flushleft}
do not include a number and do not make an entry in the table of contents.
\end{flushleft}





5





\begin{flushleft}
\newpage
For instance:
\end{flushleft}


\begin{flushleft}
\ensuremath{\backslash}section*\{Preamble\}
\end{flushleft}


\begin{flushleft}
The \ensuremath{\backslash}appendix command changes the way following sectional units are
\end{flushleft}


\begin{flushleft}
numbered. The \ensuremath{\backslash}appendix command itself generates no text and does not
\end{flushleft}


\begin{flushleft}
affect the numbering of parts.
\end{flushleft}


\begin{flushleft}
The normal use of this command is something like
\end{flushleft}


\begin{flushleft}
\ensuremath{\backslash}chapter\{A Chapter\}
\end{flushleft}


...


\begin{flushleft}
\ensuremath{\backslash}appendix
\end{flushleft}


\begin{flushleft}
\ensuremath{\backslash}chapter\{The First Appendix\}
\end{flushleft}


\begin{flushleft}
The secnumdepth counter controls printing of section numbers. The setting suppresses heading numbers at any depth $>$ level, where chapter is level
\end{flushleft}


\begin{flushleft}
zero.
\end{flushleft}


\begin{flushleft}
\ensuremath{\backslash}setcounter\{secnumdepth\}\{level\}
\end{flushleft}





4





\begin{flushleft}
Cross Reference
\end{flushleft}





\begin{flushleft}
One reason for numbering things like figures and equations is to refer the
\end{flushleft}


\begin{flushleft}
reader to them, as in \^{a}��Figure 3 for more details.\^{a}��
\end{flushleft}





4.1





\begin{flushleft}
\ensuremath{\backslash}label\{key\}
\end{flushleft}





\begin{flushleft}
A \ensuremath{\backslash}label command appearing in ordinary text assigns to key the number
\end{flushleft}


\begin{flushleft}
of the current sectional unit; one appearing inside a numbered environment
\end{flushleft}


\begin{flushleft}
assigns that number to key.
\end{flushleft}


\begin{flushleft}
A key name can consist of any sequence of letters, digits, or punctuation
\end{flushleft}


\begin{flushleft}
characters. Upper and lowercase letters are distinguished.
\end{flushleft}





6





\begin{flushleft}
\newpage
To avoid accidentally creating two labels with the same name, it is common to use labels consisting of a prefix and a suffix separated by a colon or
\end{flushleft}


\begin{flushleft}
period. Some conventionally-used prefixes:
\end{flushleft}


\begin{flushleft}
ch
\end{flushleft}


\begin{flushleft}
sec
\end{flushleft}


\begin{flushleft}
fig
\end{flushleft}


\begin{flushleft}
tab
\end{flushleft}


\begin{flushleft}
eq
\end{flushleft}





4.2





\begin{flushleft}
for chapters
\end{flushleft}


\begin{flushleft}
for lower-level sectioning commands
\end{flushleft}


\begin{flushleft}
for figures
\end{flushleft}


\begin{flushleft}
for tables
\end{flushleft}


\begin{flushleft}
for equations
\end{flushleft}





\begin{flushleft}
\ensuremath{\backslash}pageref\{key\}
\end{flushleft}





\begin{flushleft}
The \ensuremath{\backslash}pageref \{ key \} command produces the page number of the place in
\end{flushleft}


\begin{flushleft}
the text where the corresponding \ensuremath{\backslash}label \{ key \} command appears.
\end{flushleft}





4.3





\begin{flushleft}
\ensuremath{\backslash}ref\{key\}
\end{flushleft}





\begin{flushleft}
The \ensuremath{\backslash}ref command produces the number of the sectional unit, equation,
\end{flushleft}


\begin{flushleft}
footnote, figure, . . . , of the corresponding \ensuremath{\backslash}label command. It does not
\end{flushleft}


\begin{flushleft}
produce any text, such as the word \^{a}��Section\^{a}�� or \^{a}��Figure\^{a}��, just the bare number
\end{flushleft}


\begin{flushleft}
itself.
\end{flushleft}





5





\begin{flushleft}
Fonts
\end{flushleft}





5.1





\begin{flushleft}
Font Styles
\end{flushleft}





\begin{flushleft}
A few of the font styles which are useful are listed below
\end{flushleft}


\begin{flushleft}
1. \ensuremath{\backslash}textrm (Roman)
\end{flushleft}


\begin{flushleft}
2. \ensuremath{\backslash}textit (Italics)
\end{flushleft}


\begin{flushleft}
3. \ensuremath{\backslash}textbf (Bold)
\end{flushleft}


\begin{flushleft}
4. \ensuremath{\backslash}emph (Emphasis)
\end{flushleft}


\begin{flushleft}
5. \ensuremath{\backslash}texttt (Typewriter)
\end{flushleft}


\begin{flushleft}
6. \ensuremath{\backslash}textnormal (Normal font)
\end{flushleft}


7





\begin{flushleft}
\newpage
Example to output text in bold we can type
\end{flushleft}


\begin{flushleft}
\ensuremath{\backslash}textbf\{anything you type inside the curly braces will be outputed
\end{flushleft}


\begin{flushleft}
in bold\}
\end{flushleft}





5.2





\begin{flushleft}
Font Sizes
\end{flushleft}





\begin{flushleft}
The following standard type size commands are supported by LaTeX.
\end{flushleft}


\begin{flushleft}
Table 1: Font Sizes
\end{flushleft}


\begin{flushleft}
Command
\end{flushleft}





\begin{flushleft}
\ensuremath{\backslash}Large
\end{flushleft}





\begin{flushleft}
10pt
\end{flushleft}


5


7


7


9


10


12


14.4





\begin{flushleft}
\ensuremath{\backslash}LARGE
\end{flushleft}





17.28 17.28 20.74





\begin{flushleft}
\ensuremath{\backslash}tiny
\end{flushleft}





\begin{flushleft}
\ensuremath{\backslash}scriptsize
\end{flushleft}





\begin{flushleft}
\ensuremath{\backslash}footnotesize
\end{flushleft}





\begin{flushleft}
\ensuremath{\backslash}small
\end{flushleft}





\begin{flushleft}
\ensuremath{\backslash}normalize(default)
\end{flushleft}





\begin{flushleft}
\ensuremath{\backslash}large
\end{flushleft}





\begin{flushleft}
\ensuremath{\backslash}huge
\end{flushleft}


\begin{flushleft}
\ensuremath{\backslash}Huge
\end{flushleft}





\begin{flushleft}
11pt 12pt
\end{flushleft}


6


6


8


9


8


8


10


10.95


10.95


12


12


14.4


14.4 17.28





20.74 20.74 24.88


24.88 24.88 24.88





\begin{flushleft}
The commands as listed here are declaration forms. The scope of the
\end{flushleft}


\begin{flushleft}
declaration form lasts until the next type style command or the end of the
\end{flushleft}


\begin{flushleft}
current group.
\end{flushleft}





6





\begin{flushleft}
Images
\end{flushleft}





\begin{flushleft}
To insert an image we first need to include a package called graphicx after
\end{flushleft}


\begin{flushleft}
the documentclass as mentioned in section 2.1. To insert the image into the
\end{flushleft}


\begin{flushleft}
PDF we have to use the following commands
\end{flushleft}





8





\begin{flushleft}
\newpage
\ensuremath{\backslash}begin\{figure\}[h]
\end{flushleft}


\begin{flushleft}
\ensuremath{\backslash}centering
\end{flushleft}


\begin{flushleft}
\ensuremath{\backslash}caption\{Example Picture Created Using Dia\}
\end{flushleft}


\begin{flushleft}
\ensuremath{\backslash}includegraphics[scale=0.7] \{img.png\}
\end{flushleft}


\begin{flushleft}
\ensuremath{\backslash}end\{figure\}
\end{flushleft}





\begin{flushleft}
Figures are objects that are not part of the normal text, and are instead
\end{flushleft}


\begin{flushleft}
floated to a convenient place, such as the top of a page. Figures will not be
\end{flushleft}


\begin{flushleft}
split between two pages.
\end{flushleft}





\begin{flushleft}
Figure 1: Example Picture Created Using Dia
\end{flushleft}





7


7.1





\begin{flushleft}
Tables
\end{flushleft}


\begin{flushleft}
Multi Row Tables
\end{flushleft}





\begin{flushleft}
To combine rows the package multirow must be imported with
\end{flushleft}


\begin{flushleft}
in your preamble, then you can use the \ensuremath{\backslash}multirow command in your document:
\end{flushleft}


\begin{flushleft}
\ensuremath{\backslash}usepackage\{multirow\}
\end{flushleft}


\begin{flushleft}
The table below includes mathematical notations, which you can produce
\end{flushleft}


\begin{flushleft}
by embedding the experession in \$ \$ delimiters. For subscript, use underscore
\end{flushleft}


\begin{flushleft}
and for superscript, use carrot.
\end{flushleft}





9





\begin{flushleft}
\newpage
Table 2: Example of Multi Row Table
\end{flushleft}


\begin{flushleft}
Algorithms
\end{flushleft}


\begin{flushleft}
Bubble Sort
\end{flushleft}





\begin{flushleft}
Merge Sort
\end{flushleft}





\begin{flushleft}
Quick Sort
\end{flushleft}





7.2





\begin{flushleft}
Time Complexity
\end{flushleft}


\begin{flushleft}
Best Case : O(n)
\end{flushleft}


\begin{flushleft}
Average Case : O(n2 )
\end{flushleft}


\begin{flushleft}
Worst Case : O(n2 )
\end{flushleft}


\begin{flushleft}
Best Case : O(n log2 (n))
\end{flushleft}


\begin{flushleft}
Average Case : O(n log2 (n))
\end{flushleft}


\begin{flushleft}
Worst Case : O(n log2 (n))
\end{flushleft}


\begin{flushleft}
Best Case : O(n log2 (n))
\end{flushleft}


\begin{flushleft}
Average Case : O(n log2 (n))
\end{flushleft}


\begin{flushleft}
Worst Case : O(n2 )
\end{flushleft}





\begin{flushleft}
Table of Figures
\end{flushleft}


\begin{flushleft}
Table 3: Example of Table of Figures
\end{flushleft}





10





\newpage
8





\begin{flushleft}
Conclusion
\end{flushleft}





\begin{flushleft}
Thus by using LaTeX we can create reports, articles etc on the fly without
\end{flushleft}


\begin{flushleft}
worrying about the alignment, typeset etc making it very productive.LaTeX
\end{flushleft}


\begin{flushleft}
has become a popular tools among students, teachers, research scholars etc
\end{flushleft}


\begin{flushleft}
as it is a free software available on any Linux platform.
\end{flushleft}





11





\newpage



\end{document}
